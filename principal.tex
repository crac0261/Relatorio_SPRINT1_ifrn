% www.overleaf.com

%cuidado ao mexer no arquivo configuracoes
%nao tente compilar de dentro dele, não irá funcionar
%caso faça isso, volte para este arquivo e tente novamente
\include{configuracoes}



% ---
% compila o indice
% ---
\makeindex
% ---



% ---
% Informações de dados para CAPA e FOLHA DE ROSTO
% ---
\titulo{RELATÓRIO SPRINT 1 SOBRE: \\                                          Solução de baixo custo utilizando firmwares baseados em Linux para roteadores remanufaturados}
%\autor{Caio Rafael Alexandre Camilo \and Renner de Oliveira Torres Junior}
\author{
  CAIO RAFAEL ALEXANDRE CAMILO \\
  RENNER DE OLIVEIRA TORRES 
}

\local{Natal/RN}
%Sempre adicione a titulação do professor
%Ele pode ficar bem zangado se for esquecido
\orientador{Prof. Dr. Carlos Gustavo Araújo da Rocha}
\primeiroMembroBanca{Prof. Me. Percivaldo Abrupto}{Examinador interno - IFRN}
\segundoMembroBanca{Prof. Dr. Agrônio Morolov}{Examinador interno - IFRN}
\dataApresentacao{21 de Novembro de 2023}
\data{2023}
\newcommand{\nomeInstituicao}{INSTITUTO FEDERAL DE EDUCAÇÃO, CIÊNCIA E TECNOLOGIA DO RIO GRANDE DO NORTE}
\instituicao{%
  \nomeInstituicao -- IFRN
  \par
  Redes de Computadores}
\tipotrabalho{Relatório técnico}
% O preambulo deve conter o tipo do trabalho, o objetivo, 
% o nome da instituição e a área de concentração 
\preambulo{Relatório de pesquisa da disciplina do projeto integrador apresentado ao Curso Superior de Tecnologia Redes de Computadores do Instituto Federal de Educação, Ciência e Tecnologia do Rio Grande do
Norte, Campus Natal Central.}
% ---



% ----
% Início do documento
% ----
\begin{document}

% Retira espaço extra obsoleto entre as frases.
\frenchspacing 

%esse include precisava ficar depois do \begin{document}
\include{renews}


% ---

% ---
% inserir o sumario
% ---
\pdfbookmark[0]{\contentsname}{toc}
\tableofcontents*
\cleardoublepage
% ---


% ----------------------------------------------------------
% ELEMENTOS TEXTUAIS
% ----------------------------------------------------------
\textual

% ----------------------------------------------------------
% Introdução
% ----------------------------------------------------------
\secao{Objetivo Geral SPRINT 1}


Você deve definir um objetivo simples, preferencialmente que a solução do problema seja o objetivo e não a construção do sistema.
Ex:
O objetivo deste trabalho é solucionar o problema de superlotação na empresa X.

Uma vez analisados os problemas, chegou-se a conclusão de que o problema pode ser solucionado ou amenizado com um aplicativo Y que irá fazer Z coisas, assim diminuindo o problema de superlotação.

OBS: Defina um objetivo que você realmente poderá cumprir e que poderá ser verificado, por exemplo, ``O objetivo deste trabalho é melhorar o aprendizado da linguagem X'', ao final você vai conseguir provar que você melhorou o aprendizado? Você aplicará testes nos alunos para verificar isso? Se não, então é melhor escrever o objetivo de uma forma diferente

% ----------------------------------------------------------
% Capitulo com exemplos de comandos inseridos de arquivo externo 
% ----------------------------------------------------------
\subsecao{Objetivos Específicos SPRINT 1}


Aqui você irá listar alguns objetivos específicos, que seriam pontos que você precisa percorrer para chegar ao seu objetivo final. Obs, não listar objetivos muito óbvios como, ``Realizar levantamento bibliográfico'' e nem amarrar tanto  a solução do problema, ex: Se o problema for superlotação ele pode ser resolvido de várias formas, o aplicativo é apenas uma delas, então um objetivo ruim seria: "Construir o aplicativo Y"

Exemplos:

\begin{itemize}
\item Entender o funcionamento da empresa X
\item Elaborar uma solução, ou projeto
\item Construir o projeto
\item Implantar a solução
\end{itemize}






\secao{ISSUES}

\label{ref_teorico}


Ex.: Segundo \citeonline[p. 20]{doneda2006privacidade}.

Sobrenome do autor, fora do texto, nos parênteses – Minúsculos.
Ex.: \cite[p. 20]{paulino2006analise}.

Citação de páginas web, para fazer essas citações é preciso construir o código que vai para o bib, para isso é preciso descobrir o ano de publicação e autor. Muitas vezes o nome do autor vem na própria publicação, o nome do site só poderá ser usado se o nome do autor não constar na página. O ano de publicação também pode ser encontrado na página, porém é mais comum que não exista essa informação, então você pode procurar pelo ano em que o Google indexou a página. Basta fazer uma pesquisa no Google que encontre a página e utilizar junto o filtro de datas, com isso o Google irá mostrar a data de indexação junto com os resultados. Exemplo da citação online \cite{refABNTSite}.


Pode notar que tudo que é citado aparece automaticamente no final do documento em referências.

%\subsecao{Tecnologias utilizadas}

%Aqui você pode listar as tecnologias utilizadas, uma seção para cada uma. É importantíssimo destacar o motivo da escolha de cada tecnologia.


\secao{Documentos}
\label{sec:metodologia}

Na metodologia você irá descrever todo o passo a passo que você realizou para construir o seu projeto.
Ex: Primeiramente foi necessário fazer um levantamento bibliográfico utilizando as bases de dados do Google \textit{Scholar} e a palavra-chave "COVID-19". Após este levantamento bibliográfico foram feitas reuniões com o cliente, a partir dos resultados das reuniões foram gerados gráficos.

Não esqueça de explicar detalhadamente como foi a metodologia de desenvolvimento, se foi Scrum, Xp, OpenUp, ou se não foi nenhuma delas e sim algo baseado em princípios ágeis, que é o mais comum.

É interessante construir algum tipo de cronograma.



%\subsecao{Desenvolvimento}
%\label{desenvolvimento}

%Aqui você irá mostrar o seu trabalho.

%\subsubsecao{Seção Terciária}
%\label{outrasecao}

%Pode ser dividido em subseções

%\subsubsubsecao{Seção quaternária}

%Ou subseções das subseções

%Em seguida um exemplo de como fazer uma listagem de itens.

%\noindent Descrever brevemente:
%\begin{itemize}
%\item A lei;
%\item Objetivo;
%\item Carga horária;
%\item Jornada de trabalho.

%\end{itemize}

%Este parágrafo contém um exemplo de nota de rodapé\footnote{A nota de rodapé serve para explicar algum termo no rodapé da página, de forma que a leitura não é interrompida pela explicação}.

%Toda figura ou tabela deve ser referenciada e EXPLICADA, dizendo ``A Figura X representa tal coisa, e isso funciona assim por causa disso e daquilo, lembrar de sempre escrever Figura com F maiúsculo''. Aqui também está o exemplo de como se deve usar as aspas. A Figura \ref{figura:qualquernome} é um exemplo de inclusão de imagem.

%normalmente o latex ira procurar o melhor lugar no documento para encaixar a imagem/tabela, mas isso pode atrapalhar a leitura
%\begin{figure}[h]% o h faz com que ele procure um bom lugar para a imagem, mas isso pode bagunçar em alguns casos, use o H (maiúsculo) para forçar ela a ficar no local determinado
    %\caption{Exemplo de imagem}
    %\centering
   % \includegraphics[width=5cm]{imagens/grafico.png}
    %\label{figura:qualquernome}
    %\fonteimagem{Fonte: Elaborado pelo autor}
%\end{figure}


%Exemplo com duas imagens a Figura \ref{fig:a} é a figura da esquerda e a Figura \ref{fig:b} é a figura da direita.

\begin{figure}[H]
     \centering
     \caption{Telas de Nova renovação}
     \begin{subfigure}[t]{0.40\textwidth}
         \centering
        \includegraphics[width=\textwidth]{imagens/grafico.png}
         \caption{Selecionar paciente cadastrado}%
         \label{fig:a}
     \end{subfigure} %nao pode ter espaço aqui, aí fica uma ao lado da outra
     \begin{subfigure}[t]{0.40\textwidth}
         \centering
         \includegraphics[width=\textwidth]{imagens/grafico.png}
         \caption{Formulário de nova renovação}%
         \label{fig:b}
     \end{subfigure}
     
    \fonteimagem{Fonte: Elaborado pelo autor}
    \label{fig:13}
\end{figure}

Dica: é comum o desenvolvimento de protótipos, quando assim for, coloque-os todos dentro de um apêndice e indique: os protótipos construídos estão disponíveis no Apêndice \ref{appendix:prototipos} deste documento. Os anexos são para documentos que você não produziu, mas que acha importante para o texto.

\newpage

Há uma diferença entre tabelas e quadros, tabelas não possuem linhas verticais, quadros possuem, no mais são muito semelhantes, inclusive a apresentação no texto. O Quadro \ref{quad:ex} é um exemplo. Sugiro o uso do site \url{https://www.tablesgenerator.com/} para gerar o código:



\begin{quadro}
\caption{Exemplo de quadro}
\label{quad:ex}
\centering
\begin{tabular}{|l|c|c|c|}
\hline
\multicolumn{1}{|c|}{} & \multicolumn{1}{l|}{\textbf{Qtd}} & \multicolumn{1}{l|}{\textbf{Valor R\$}} & \multicolumn{1}{l|}{\textbf{Tipo}} \\ \hline
\rowcolor[HTML]{FFCC67} 
\textbf{Produto A}     & 2                                 & 22                                      & C22                                \\ \hline
\textbf{Produto C}     & 3                                 & 33                                      & C88                                \\ \hline
\textbf{Produto D}     & 4                                 & 11                                      & c40                                \\ \hline
\end{tabular}
\fonteimagem{Fonte: Elaborado pelo autor}
\end{quadro}


A Tabela \ref{tab:ex} é um exemplo de tabela, não possui linhas verticais.



\begin{table}[h]% o h força a imagem/tabela a ficar neste local
\caption{Minha tabela}
\label{tab:ex}
\centering
\begin{tabular}{lccc}
\hline
\multicolumn{1}{c}{} & \multicolumn{1}{l}{\textbf{Qtd}} & \multicolumn{1}{l}{\textbf{Valor R\$}} & \multicolumn{1}{l}{\textbf{Tipo}} \\ \hline
\rowcolor[HTML]{FFCC67} 
\textbf{Produto A}     & 2                                 & 22                                      & C22                                \\ \hline
\textbf{Produto C}     & 3                                 & 33                                      & C88                                \\ \hline
\textbf{Produto D}     & 4                                 & 11                                      & c40                                \\ \hline
\end{tabular}
\fonteimagem{Fonte: Elaborado pelo autor}
\end{table}

Também é possível incluir fórmulas matemáticas, caso esteja com dificuldade para construir as equações, construa a equação em https://pt.symbolab.com/solver/equation-calculator, copie e cole aqui, obs, a equação deve estar entre \$equação\$ ou \$\$equação\$\$ (destacada). Outro detalhe é que não pode haver espaços entre o sinal \$ e a equação, exemplo:

$$x^n + y^n = z^n$$

Para incluir o sinal de porcentagem (\%) e outros sinais(\$ \# \^ \space \& \_ \{ \} \~ \space ) é necessário escapar o sinal com a barra invertida ($\backslash$).

O Código Fonte \ref{lst:exemplocodigo1} é um exemplo de como colocar código no documento.


%exemplo de código fonte, as configurações estão no arquivo packages.tex
\lstinputlisting[language=PHP, 
caption=Uma função em PHP
,label=lst:exemplocodigo1]{trechos_codigo/php.m}


\secao{Commits}

Por fim, bahshahakjsdhkajhsdkjahskdhaksjhdakjhsdkjahsdkjhakjshdkajhsdkj

OBS: NUNCA diga que não fez algo porque não deu tempo, se não daria tempo por que foi planejado? Se foi planejado e não deu tempo, você não administrou corretamente o projeto. É ainda pior se for algo muito simples.

%\subsecao{\uppercase{Trabalhos futuros}}
%Evite colocar aqui coisas que deveriam ter sido feitas no seu trabalho.
%\begin{itemize}
%\item Pense em algo bem mais avançado para o seu trabalho.
%\item E sugira em tópicos.
%\end{itemize}


\secao{Activity}

Por fim,bahshahakjsdhkajhsdkjahskdhaksjhdakjhsdkjahsdkjhakjshdkajhsdkj

OBS: NUNCA diga que não fez algo porque não deu tempo, se não daria tempo por que foi planejado? Se foi planejado e não deu tempo, você não administrou corretamente o projeto. É ainda pior se for algo muito simples.

%\subsecao{\uppercase{Trabalhos futuros}}
%Evite colocar aqui coisas que deveriam ter sido feitas no seu trabalho.
%\begin{itemize}
%\item Pense em algo bem mais avançado para o seu trabalho.
%\item E sugira em tópicos.
%\end{itemize}

\secao{Reuniões}

\secao{Códigos e testes Realizados}

% ----------------------------------------------------------
% ELEMENTOS PÓS-TEXTUAIS
% ----------------------------------------------------------
\postextual


\newpage


% ----------------------------------------------------------
% Referências bibliográficas
% ----------------------------------------------------------
\renewcommand{\bibname}{\textbf{REFERÊNCIAS}}
%v 2.3
\bibliographystyle{abntex2-cite-min}
\bibliography{referencias}

% ----------------------------------------------------------
% Glossário
% ----------------------------------------------------------
%
% Consulte o manual da classe abntex2 para orientações sobre o glossário.
%
%\glossary

% ----------------------------------------------------------
% Apêndices
% ----------------------------------------------------------
%
%% ---
%% Inicia os apêndices
%% ---
%\begin{apendicesenv}
%
%% Imprime uma página indicando o início dos apêndices
%\partapendices
%
%% ----------------------------------------------------------
%\chapter{Protótipos do sistema}
%\label{appendix:prototipos}
%% ----------------------------------------------------------
%
%\begin{figure}[h]
   % \caption{Exemplo de imagem A}
  %  \centering
   % \includegraphics[width=5cm]{imagens/grafico.png}
    %\label{figura:qualquernome2}
   % \fonteimagem{Fonte: Elaborado pelo autor}
%\end{figure}

%begin{figure}[h]
    %\caption{Exemplo de imagem B}
   % \centering
    %\includegraphics[width=5cm]{imagens/grafico.png}
    %\label{figura:qualquernome3}
    %\fonteimagem{Fonte: Elaborado pelo autor}
%\end{figure}
%
%% ----------------------------------------------------------
%\chapter{Nullam elementum urna vel imperdiet sodales elit ipsum pharetra ligula
%ac pretium ante justo a nulla curabitur tristique arcu eu metus}
%% ----------------------------------------------------------
%\lipsum[55-57]
%
%\end{apendicesenv}
%% ---
%
%
%% ----------------------------------------------------------
%% Anexos
%% ----------------------------------------------------------
%
%% ---
%% Inicia os anexos
%% ---
\begin{anexosenv}
%
%% Imprime uma página indicando o início dos anexos
\partanexos
%
%% ---
\chapter{Comprovante de orientação}
%% ---
\lipsum[30]
%
%% ---
%\chapter{Cras non urna sed feugiat cum sociis natoque penatibus et magnis dis parturient montes nascetur ridiculus mus}
%% ---
%
\lipsum[31]
%


\end{anexosenv}
%
%%---------------------------------------------------------------------
%% INDICE REMISSIVO
%%---------------------------------------------------------------------
%
%\phantompart
%
%\printindex

\end{document}
