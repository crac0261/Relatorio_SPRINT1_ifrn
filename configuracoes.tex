
\documentclass[
    article,
	% -- opções da classe memoir --
	12pt,				% 
	openright,			% capítulos começam em pág ímpar (insere página vazia caso preciso)
	oneside,
	%twoside,			% para impressão em verso e anverso. Oposto a oneside
	a4paper,			% tamanho do papel. 
	% -- opções da classe abntex2 --
	%chapter=TITLE,		% títulos de capítulos convertidos em letras maiúsculas
	section=TITLE,		% títulos de seções convertidos em letras maiúsculas
	subsection=TITLE,	% títulos de subseções convertidos em letras maiúsculas
	%subsubsection=TITLE,% títulos de subsubseções convertidos em letras maiúsculas
	% -- opções do pacote babel --
	english,			% idioma adicional para hifenização
	french,				% idioma adicional para hifenização
	spanish,			% idioma adicional para hifenização
	brazil,				% o último idioma é o principal do documento
	]{abntex2}
	

%atualizacao em 31-10-2022
\usepackage{abntex2/IFRN}

\usepackage{tgtermes}
\usepackage{fancyhdr}
\pagestyle{fancy}
\pagestyle{fancyplain}
\fancyhf{}
\lhead{}
\rhead{\thepage}
\rfoot{}
\renewcommand{\headrulewidth}{0pt}
% ---
% PACOTES
% ---

% ---
% Pacotes fundamentais 
% ---
\usepackage{lmodern}			% Usa a fonte Latin Modern
\usepackage[T1]{fontenc}		% Selecao de codigos de fonte.
\usepackage[utf8]{inputenc}		% Codificacao do documento (conversão automática dos acentos)
\usepackage{indentfirst}		% Indenta o primeiro parágrafo de cada seção.
\usepackage{color}				% Controle das cores
\usepackage{graphicx}			% Inclusão de gráficos
\usepackage{microtype} 			% para melhorias de justificação
\usepackage{float}
% ---
\usepackage{caption}
\captionsetup[figure]{font=normalsize}
%v 2.3
\captionsetup[table]{font=normalsize}
\captionsetup[quadro]{font=normalsize}
%v 2.3
\usepackage{subcaption}
%\usepackage[export]{adjustbox}
\usepackage{graphbox,graphicx}
\usepackage{pdfpages}

% ---
% Pacotes adicionais, usados no anexo do modelo de folha de identificação
% ---
\usepackage{multicol}
\usepackage{multirow}
% ---
\usepackage[font=footnotesize]{caption}%textfont=bf,labelfont=bf,
\usepackage[table,xcdraw]{xcolor}


% ---
% Pacotes adicionais, usados apenas no âmbito do Modelo Canônico do abnteX2
% ---
\usepackage{lipsum}				% para geração de dummy text
% ---
\usepackage{titlesec}
\usepackage[left=3cm,top=3cm,right=2cm,bottom=2cm]{geometry}



% ---
% Pacotes de citações
% ---
\usepackage[brazilian,hyperpageref]{backref} % Paginas com as citações na bibl


\usepackage[alf,
            %versalete,
            abnt-emphasize = bf, % destaca o titulo em negrito;
            %abnt-etal-list = 3, % trabalhos com mais de 3 autores recebem et al.,;
            %abnt-etal-text = it, % escreve o et al., em italico;
            %abnt-and-type = &, % usa o carater '&' no lugar de 'e' para mais de um autor;
            %abnt-last-names = abnt, % trata sobrenomes 'estritamente' conforme a ABNT; e
            %abnt-repeated-author-omit = yes % autores com + de uma entrada recebem '____.'
]{abntex2cite}
% --- 
% CONFIGURAÇÕES DE PACOTES
% --- 

\usepackage{url6023}


% ---
% Configurações do pacote backref
% Usado sem a opção hyperpageref de backref
\renewcommand{\backrefpagesname}{Citado na(s) página(s):~}
% Texto padrão antes do número das páginas
\renewcommand{\backref}{}
% Define os textos da citação
\renewcommand*{\backrefalt}[4]{
	\ifcase #1 %
		Nenhuma citação no texto.%
	\or
		Citado na página #2.%
	\else
		Citado #1 vezes nas páginas #2.%
	\fi}%
% ---




% ---
% Configurações de aparência do PDF final

% alterando o aspecto da cor azul
\definecolor{blue}{RGB}{41,5,195}

% informações do PDF
\makeatletter
\hypersetup{
     	%pagebackref=true,
		pdftitle={\@title}, 
		pdfauthor={\@author},
    	pdfsubject={\imprimirpreambulo},
	    pdfcreator={LaTeX with abnTeX2},
		pdfkeywords={abnt}{latex}{abntex}{abntex2}{relatório técnico}, 
		colorlinks=true,       		% false: boxed links; true: colored links
    	linkcolor=black,          	% color of internal links
    	citecolor=black,        		% color of links to bibliography
    	filecolor=magenta,      		% color of file links
		urlcolor=blue,
		bookmarksdepth=4
}
\makeatother
% --- 

% --- 
% Espaçamentos entre linhas e parágrafos 
% --- 

% O tamanho do parágrafo é dado por:
\setlength{\parindent}{1.3cm}

% Controle do espaçamento entre um parágrafo e outro:
\setlength{\parskip}{0.2cm}  % tente também \onelineskip


%atualizacao em 31-03-2021

\renewcommand{\imprimircapa}{%
\begin{capa}%
    \center
        \centering \nomeInstituicao 
        
        \vspace*{3cm}
        
        {\ABNTEXchapterfont\normalsize\imprimirautor} 
        
        \vspace*{4cm}
        
        {\ABNTEXchapterfont\bfseries\normalsize\imprimirtitulo} 
        \vspace*{\fill}
        
        {\normalsize\imprimirlocal} 
            \par
        {\normalsize\imprimirdata}
        
    \vspace*{1cm}
    
\end{capa}
}



\renewcommand{\imprimirfolhaderosto}{%
\begin{folhadeaprovacao}%
    \center
        
        {\ABNTEXchapterfont\normalsize\imprimirautor} 
        
        \vspace*{4cm}
        
        {\ABNTEXchapterfont\bfseries\normalsize\imprimirtitulo} 
        \vspace*{2cm}
        
        \hspace{.45\textwidth}
        \begin{minipage}{.5\textwidth}
            \singlespacing
            \imprimirpreambulo \\
            \\
            Orientador: \imprimirorientador
        \end{minipage}%
        
        \vspace*{\fill}
        
        {\normalsize\imprimirlocal} 
            \par
        {\normalsize\imprimirdata}
        
    \vspace*{1cm}
    
    \newpage
    
\end{folhadeaprovacao}
}

%atualizacao em 31-03-2021





%atualização em 31/10/2022
\usepackage{listings}

% ----------------------------------------------------------
% PERSONALIZAÇÃO DE CORES
% ----------------------------------------------------------
\definecolor{cinza}{HTML}{FCF8F8}
\definecolor{blue}{RGB}{41,5,195}
\definecolor{gray}{rgb}{.4,.4,.4}
\definecolor{gray}{rgb}{.4,.4,.4}
\definecolor{pblue}{rgb}{0.13,0.13,1}
\definecolor{pgreen}{rgb}{0,0.5,0}
\definecolor{pred}{rgb}{0.9,0,0}
\definecolor{pgrey}{rgb}{0.46,0.45,0.48}
\definecolor{lightgray}{rgb}{0.95, 0.95, 0.96}
\definecolor{whitesmoke}{rgb}{0.96, 0.96, 0.96}
\definecolor{javared}{rgb}{0.6,0,0} % for strings
\definecolor{javagreen}{rgb}{0.25,0.5,0.35} % comments
\definecolor{javapurple}{rgb}{0.5,0,0.35} % keywords
\definecolor{javadocblue}{rgb}{0.25,0.35,0.75} % javadoc
\definecolor{meucinza}{rgb}{0.5, 0.5, 0.5}
%\definecolor{lightgray}{gray}{0.9}

% define formato e estilo dos elementos do tipo Codigo Fonte
\lstset{language=PHP,
basicstyle=\ttfamily\scriptsize,
%basicstyle=\ttfamily,
keywordstyle=\color{javapurple}\bfseries,
stringstyle=\color{pblue},
commentstyle=\color{javagreen},
morecomment=[s][\color{javadocblue}]{/**}{*/},
morecomment=[s][\color{gray}]{@}{\ },
numbers=left,
numberstyle=\tiny\color{black},
backgroundcolor=\color{cinza},
stepnumber=2,
numbersep=8pt,
xleftmargin=14pt,
tabsize=4,
showspaces=false,
showstringspaces=false,
breaklines=true,}

%%%%%%%%%%%%%%%%%%%%%%%%%%%%%%%%%%



\newenvironment{ficha}{\textbf{Ficha catalográfica}}

\providecommand{\imprimirDataApresentacao}{}
\newcommand{\dataApresentacao}[1]{\renewcommand{\imprimirDataApresentacao}{#1}}

\providecommand{\imprimirPrimeiroMembroBanca}{}
\providecommand{\imprimirInstituicaoPrimeiroMembroBanca}{}
\newcommand{\primeiroMembroBanca}[2]{
    \renewcommand{\imprimirPrimeiroMembroBanca}{#1}
    \renewcommand{\imprimirInstituicaoPrimeiroMembroBanca}{#2}
}


\providecommand{\imprimirSegundoMembroBanca}{}
\providecommand{\imprimirInstituicaoSegundoMembroBanca}{}
\newcommand{\segundoMembroBanca}[2]{
    \renewcommand{\imprimirSegundoMembroBanca}{#1}
    \renewcommand{\imprimirInstituicaoSegundoMembroBanca}{#2}
}



\newcommand{\folhaaprovacao}{
%
\begin{folhadeaprovacao}

  \begin{center}
    {\ABNTEXchapterfont\large\imprimirautor}

    \vspace*{\fill}\vspace*{\fill}
    \begin{center}
      \ABNTEXchapterfont\bfseries\normalsize\imprimirtitulo
    \end{center}
    \vspace*{2cm}
    
    \hspace{.45\textwidth}
    \begin{minipage}{.5\textwidth}
        \imprimirpreambulo
    \end{minipage}%
    \vspace*{\fill}
   \end{center}
        
   Trabalho aprovado. \imprimirlocal, \imprimirDataApresentacao.
   \newline
   \newline
   \centering BANCA EXAMINADORA


   %sempre manter o título acadêmico dos professores
   \assinatura{\textbf{\imprimirorientador} \\ Orientador - IFRN} 
   \assinatura{\textbf{\imprimirPrimeiroMembroBanca} \\ \imprimirInstituicaoPrimeiroMembroBanca}
   \assinatura{\textbf{\imprimirSegundoMembroBanca} \\ \imprimirInstituicaoSegundoMembroBanca}

    \newpage
\end{folhadeaprovacao}
}


%%%%%%%%%%%%%%%%%%%%%%%%%%%%%%%%%%%%%%%%
% \newcommand{\espacofichacatalografica}{
% \begin{ficha}
%     \centering Aqui entrará a ficha catalográfica
%     \newpage
% \end{ficha}
% }